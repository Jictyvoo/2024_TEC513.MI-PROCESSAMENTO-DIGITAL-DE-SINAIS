
\documentclass[conference]{IEEEtran}
\IEEEoverridecommandlockouts
% The preceding line is only needed to identify funding in the first footnote. If that is unneeded, please comment it out.
\usepackage{cite}
\usepackage{amsmath,amssymb,amsfonts}
\usepackage{algorithmic}
\usepackage{graphicx}
\usepackage{textcomp}
\usepackage{xcolor}
\usepackage{float}
\usepackage{listings}
\usepackage[portuguese]{babel}

\usepackage{xcolor}
\usepackage[hashEnumerators,smartEllipses,fencedCode]{markdown} % This package will help to write all codes
\def\markdownOptionOutputDir{./out}

\definecolor{codegreen}{rgb}{0,0.6,0}
\definecolor{codegray}{rgb}{0.5,0.5,0.5}
\definecolor{codepurple}{rgb}{0.58,0,0.82}
\definecolor{backcolour}{rgb}{0.95,0.95,0.92}

\lstdefinestyle{mystyle}{
    backgroundcolor=\color{backcolour},   
    commentstyle=\color{codegreen},
    keywordstyle=\color{magenta},
    numberstyle=\tiny\color{codegray},
    stringstyle=\color{codepurple},
    basicstyle=\ttfamily\footnotesize,
    breakatwhitespace=false,         
    breaklines=true,                 
    captionpos=b,                    
    keepspaces=true,                 
    numbers=left,                    
    numbersep=5pt,                  
    showspaces=false,                
    showstringspaces=false,
    showtabs=false,                  
    tabsize=2
}

\lstset{style=mystyle}

\def\BibTeX{{\rm B\kern-.05em{\sc i\kern-.025em b}\kern-.08em
    T\kern-.1667em\lower.7ex\hbox{E}\kern-.125emX}}

\begin{document}

   \makeatletter
    \newcommand{\linebreakand}{%
      \end{@IEEEauthorhalign}
      \hfill\mbox{}\par
      \mbox{}\hfill\begin{@IEEEauthorhalign}
    }
    \makeatother

\title{Desenvolvimento e análise do algoritmo FFT\\
}

\author{
\IEEEauthorblockN{1\textsuperscript{st} João Victor Oliveira Couto}
\IEEEauthorblockA{\textit{Estudante de Eng. de Computação} \\
\textit{Universidade Estadual de Feira de Santana}\\
Feira de Santana, Bahia, Brasil \\
contact@jityvoo.dev}
\IEEEauthorblockN{}
\IEEEauthorblockA{}
}

\maketitle

\begin{abstract}
    Neste trabalho, foram desenvolvidos dois algoritmos para calcular a Transformada de Fourier Discreta (DFT) e Fast Fourier Transform (FFT) com dizimação na frequência. O primeiro algoritmo é uma implementação básica da DFT, enquanto o segundo utiliza o algoritmo Cooley-Tukey para realizar a FFT. 

    As implementações no sinal fornecido foram testadas e analisadas em termos de precisão e complexidade computacional. Também é realizado o estudo do efeito do aumento do tamanho da janela na performance e resultado desses algoritmos.
\end{abstract}

\begin{IEEEkeywords}
    Discrete Fourier Transform, Fast Fourier Transform, Cooley-Tukey algorithm, signal processing, frequency domain decimation, computational complexity.
\end{IEEEkeywords}

\section{Desenvolvimento de algoritmos para calcular a DFT}
A análise experimental foi realizada gerando os sinais que correspondem à seguinte função:

\begin{align}
    x(t) = \cos{2\pi 100 t} + 3 \cos{2\pi 250 t} + \\
    5 \cos{2\pi 750 t} + 7 \cos(2\pi 1000 t)    
\end{align}

Assim sendo, para gerar o sinal, foi desenvolvida uma função chamada \textit{generateSignal}, que irá gerar um sinal com frequência de amostragem igual a $2500\, \text{Hz}$.

\markdownInput{03_experimental_analysis/signal_gen.md}


\section{Análise experimental}
A análise experimental foi realizada gerando os sinais que correspondem à seguinte função:

\begin{align}
    x(t) = \cos{2\pi 100 t} + 3 \cos{2\pi 250 t} + \\
    5 \cos{2\pi 750 t} + 7 \cos(2\pi 1000 t)    
\end{align}

Assim sendo, para gerar o sinal, foi desenvolvida uma função chamada \textit{generateSignal}, que irá gerar um sinal com frequência de amostragem igual a $2500\, \text{Hz}$.

\markdownInput{03_experimental_analysis/signal_gen.md}


\subsection*{Amostre o sinal}
Amostre o sinal  a um período de amostragem e determine os seguintes:

\begin{enumerate}[(a)]
\item \textbf{(1,0 pontos)} Calcule a DFT e FFT do sinal amostrado com uma janela de somente 32 amostras. É possível observar no espectro as senóides?
\item \textbf{(1,0 pontos)} Aumente o comprimento do item anterior para 64 amostras, aumentando 32 zeros à direita das amostras originais. Calcule a DFT e FFT. Compare com o item anterior e comente
seus resultados.
\item \textbf{(1,0 pontos)} Calcule a DFT e FFT usando uma janela de 64 amostras. É possivel observar no espectro as senóides?
\item \textbf{(1,0 pontos)} Aumente o comprimento do item anterior para 128 amostras, aumentando 64 zeros à direita das amostras originais. Calcule a DFT e FFT. Compare com o item anterior e comente seus resultados.
\item \textbf{(1,0 pontos)} E assim por diante, repita os ítens (a) e (b) para 256, 512 e 1024 amostras do sinal.
\item \textbf{(1,0 pontos)} Monte numa tabela comparativa a quantidade de operações (produtos e somas) realizadas.
\end{enumerate}


\appendix
\section{Detalhes Adicionais}

\end{document}
