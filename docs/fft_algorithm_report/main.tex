\documentclass[conference]{IEEEtran}
\IEEEoverridecommandlockouts
% The preceding line is only needed to identify funding in the first footnote. If that is unneeded, please comment it out.
\usepackage{cite}
\usepackage{amsmath,amssymb,amsfonts}
\usepackage{algorithmic}
\usepackage{graphicx}
\usepackage{textcomp}
\usepackage{xcolor}
\usepackage{float}
\usepackage{listings}
\usepackage[portuguese]{babel}

\usepackage{xcolor}
\usepackage[hashEnumerators,smartEllipses,fencedCode]{markdown} % This package will help to write all codes
\def\markdownOptionOutputDir{./out}

\definecolor{codegreen}{rgb}{0,0.6,0}
\definecolor{codegray}{rgb}{0.5,0.5,0.5}
\definecolor{codepurple}{rgb}{0.58,0,0.82}
\definecolor{backcolour}{rgb}{0.95,0.95,0.92}

\lstdefinestyle{mystyle}{
    backgroundcolor=\color{backcolour},   
    commentstyle=\color{codegreen},
    keywordstyle=\color{magenta},
    numberstyle=\tiny\color{codegray},
    stringstyle=\color{codepurple},
    basicstyle=\ttfamily\footnotesize,
    breakatwhitespace=false,         
    breaklines=true,                 
    captionpos=b,                    
    keepspaces=true,                 
    numbers=left,                    
    numbersep=5pt,                  
    showspaces=false,                
    showstringspaces=false,
    showtabs=false,                  
    tabsize=2
}

\lstset{style=mystyle}

\def\BibTeX{{\rm B\kern-.05em{\sc i\kern-.025em b}\kern-.08em
    T\kern-.1667em\lower.7ex\hbox{E}\kern-.125emX}}

\begin{document}

   \makeatletter
    \newcommand{\linebreakand}{%
      \end{@IEEEauthorhalign}
      \hfill\mbox{}\par
      \mbox{}\hfill\begin{@IEEEauthorhalign}
    }
    \makeatother

\title{Desenvolvimento e análise comparativa dos algoritmos DFT e FFT
}

\author{
\IEEEauthorblockN{1\textsuperscript{st} João Victor Oliveira Couto}
\IEEEauthorblockA{\textit{Estudante de Eng. de Computação} \\
\textit{Universidade Estadual de Feira de Santana}\\
Feira de Santana, Bahia, Brasil \\
contact@jityvoo.dev}
\IEEEauthorblockN{}
\IEEEauthorblockA{}
}

\maketitle

\begin{abstract}
    Neste trabalho, foram desenvolvidos dois algoritmos para calcular a Transformada de Fourier Discreta (DFT) e Fast Fourier Transform (FFT) com dizimação na frequência. 

    Baseado na implementação e no sinal fornecido, foi realizado o estudo do efeito do aumento do tamanho da janela na performance e resultado desses algoritmos.
\end{abstract}

\begin{IEEEkeywords}
    Discrete Fourier Transform, Fast Fourier Transform, Cooley-Tukey algorithm, signal processing, frequency domain decimation, computational complexity.
\end{IEEEkeywords}

\section{Desenvolvimento de algoritmos para calcular a DFT}
\subsection*{1. Algoritmo DFT \textbf{(2,0 pontos)}}

\subsection*{2. Algoritmo FFT com dizimação na frequência \textbf{(2,0 pontos)}}
\markdownInput{01_algorithm_development/fft_algorithm.md}


\section{Análise experimental}
\subsection*{1. Algoritmo DFT \textbf{(2,0 pontos)}}

\subsection*{2. Algoritmo FFT com dizimação na frequência \textbf{(2,0 pontos)}}
\markdownInput{01_algorithm_development/fft_algorithm.md}


\section{Conclusão}
Conclui-se que a quantidade de amostras desempenha um papel crucial na análise discreta do sinal, uma vez que influencia diretamente a resolução espectral e a precisão na identificação das componentes de frequência.

O aumento no número de amostras resulta em uma representação mais detalhada do sinal, permitindo uma discriminação mais precisa das frequências dominantes e aprimorando a qualidade da análise espectral.


%\appendix
%\section{Detalhes Adicionais}

\end{document}
