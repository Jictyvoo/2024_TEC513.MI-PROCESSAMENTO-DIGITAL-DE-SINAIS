Amostre o sinal  a um período de amostragem e determine os seguintes:

\subsubsection*{(a) \textbf{(1,0 pontos) }}
Calcule a DFT e FFT do sinal amostrado com uma janela de somente 32 amostras. É possível observar no espectro as senóides?

\begin{figure}[H]
    \centering
    \includegraphics[width=1\linewidth]{03_experimental_analysis/plot_results/32_samples_dft_fft.png}
    \caption{DFT+FFT Aplicada a 32 amostras}
    \label{fig:signal_32samples_fft-dft}
\end{figure}

%%%%%%%%%%%%%%%%%%%%%%%%%%%%%%%%%%%%%%%%%%%%%%%%%%%%%%%%%%%%%%%%%%%%%%
\subsubsection*{(b) \textbf{(1,0 pontos)}}
Aumente o comprimento do item anterior para 64 amostras, aumentando 32 zeros à direita das amostras originais. Calcule a DFT e FFT. Compare com o item anterior e comente seus resultados.

\begin{figure}[H]
    \centering
    \includegraphics[width=1\linewidth]{03_experimental_analysis/plot_results/32_samples_dft_fft_padded.png}
    \caption{DFT+FFT Aplicada a 32 amostras com 32 zeros à direita}
    \label{fig:signal_32samples_fft-dft_padded}
\end{figure}

%%%%%%%%%%%%%%%%%%%%%%%%%%%%%%%%%%%%%%%%%%%%%%%%%%%%%%%%%%%%%%%%%%%%%%
\subsubsection*{(c) \textbf{(1,0 pontos)}}
Calcule a DFT e FFT usando uma janela de 64 amostras. É possivel observar no espectro as senóides?

%%%%%%%%%%%%%%%%%%%%%%%%%%%%%%%%%%%%%%%%%%%%%%%%%%%%%%%%%%%%%%%%%%%%%%
\subsubsection*{(d) \textbf{(1,0 pontos)}}
Aumente o comprimento do item anterior para 128 amostras, aumentando 64 zeros à direita das amostras originais. Calcule a DFT e FFT. Compare com o item anterior e comente seus resultados.

%%%%%%%%%%%%%%%%%%%%%%%%%%%%%%%%%%%%%%%%%%%%%%%%%%%%%%%%%%%%%%%%%%%%%%
\subsubsection*{(e) \textbf{(1,0 pontos)}}
E assim por diante, repita os ítens (a) e (b) para 256, 512 e 1024 amostras do sinal.

%%%%%%%%%%%%%%%%%%%%%%%%%%%%%%%%%%%%%%%%%%%%%%%%%%%%%%%%%%%%%%%%%%%%%%
\subsubsection*{(f) \textbf{(1,0 pontos)}}
Monte numa tabela comparativa a quantidade de operações (produtos e somas) realizadas.
