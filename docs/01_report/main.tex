
\documentclass[conference]{IEEEtran}
\IEEEoverridecommandlockouts
% The preceding line is only needed to identify funding in the first footnote. If that is unneeded, please comment it out.
\usepackage{cite}
\usepackage{amsmath,amssymb,amsfonts}
\usepackage{algorithmic}
\usepackage{graphicx}
\usepackage{textcomp}
\usepackage{xcolor}
\usepackage{float}

\def\BibTeX{{\rm B\kern-.05em{\sc i\kern-.025em b}\kern-.08em
    T\kern-.1667em\lower.7ex\hbox{E}\kern-.125emX}}

\begin{document}

   \makeatletter
    \newcommand{\linebreakand}{%
      \end{@IEEEauthorhalign}
      \hfill\mbox{}\par
      \mbox{}\hfill\begin{@IEEEauthorhalign}
    }
    \makeatother

\title{Problema 1: Análise Matemática Comparativa da Amostragem de Sinais Analógicos\\
}

\author{\IEEEauthorblockN{1\textsuperscript{st} Antony Araujo Oliveira}
\IEEEauthorblockA{\textit{Estudante de Eng. de Computação} \\
\textit{Universidade Estadual de Feira de Santana}\\
Feira de Santana, Bahia, Brasil \\
antony@ecomp.uefs.br}
\and
\IEEEauthorblockN{2\textsuperscript{nd} Daniel Alves Costa}
\IEEEauthorblockA{\textit{Estudante de Eng. de Computação} \\
\textit{Universidade Estadual de Feira de Santana}\\
Feira de Santana, Bahia, Brasil \\
dancosta@ecomp.uefs.br} 
\and
\IEEEauthorblockN{3\textsuperscript{rd} João Victor Oliveira Couto}
\IEEEauthorblockA{\textit{Estudante de Eng. de Computação} \\
\textit{Universidade Estadual de Feira de Santana}\\
Feira de Santana, Bahia, Brasil \\
contact@jityvoo.dev}
\and
\IEEEauthorblockN{4\textsuperscript{th} Lokisley Oliveira Pedreira}
\IEEEauthorblockA{\textit{Estudante de Eng. de Computação} \\
\textit{Universidade Estadual de Feira de Santana}\\
Feira de Santana, Bahia, Brasil \\
lokisley@hotmail.com}
\and
\IEEEauthorblockN{5\textsuperscript{th} Velder Soares}
\IEEEauthorblockA{\textit{Estudante de Eng. de Computação} \\
\textit{Universidade Estadual de Feira de Santana}\\
Feira de Santana, Bahia, Brasil \\
velder.vas@gmail.com}
\IEEEauthorblockN{}
\IEEEauthorblockA{}
}

\maketitle

\begin{abstract}
Este relatório aborda o processo de conversão de sinais analógicos em digitais, explorando conceitos como o Teorema da Amostragem e a Transformada Rápida de Fourier, essenciais para a modulação e reconstrução de sinais. Através de simulações, foram analisados três métodos de amostragem — ideal, natural e \textit{flat-top} — e seus efeitos na precisão da reconstrução do sinal, destacando a importância da escolha correta da taxa de amostragem e do uso de filtros \textit{anti-aliasing} para evitar distorções como o \textit{aliasing}. O estudo reforça que a implementação de filtros adequados e a seleção correta da frequência de amostragem são cruciais para garantir a fidelidade do sinal no processamento digital. 
\end{abstract}

\begin{IEEEkeywords}
Conversão analógico-digital, Teorema da Amostragem, Transformada de Fourier, aliasing, modulação por amplitude de pulso (PAM), filtros anti-aliasing, reconstrução de sinais e taxa de amostragem .
\end{IEEEkeywords}

\section{Introdução}
A conversão de sinais analógicos em processamento digital de sinais é fundamental em telecomunicações, redes de computadores, sistemas inteligentes, aplicações industriais, militares e governamentais. Isso se deve ao fato de que muitos sinais naturais, como grandezas físicas, são analógicos ou contínuos, e para que possam ser processados em dispositivos eletrônicos e computacionais, precisam ser convertidos para uma representação discreta. 

\begin{figure}[H]
    \centering
    \includegraphics[width=1\linewidth]{image.png}
    \caption{Processamento a tempo discreto de sinais a tempo contínuo. Fonte: SANCA (2024)}
    \label{fig:enter-label}
\end{figure}
O processo de conversão analógico-digital (A/D) envolve três etapas principais: amostragem, quantização e codificação, sendo a amostragem o primeiro passo, onde valores discretos são capturados de um sinal originalmente contínuo no tempo.

A amostragem é crucial porque é nesse estágio que o sinal contínuo é transformado em uma sequência de pontos discretos que representará o sinal digitalmente. A qualidade dessa conversão afeta diretamente a precisão com que o sinal original pode ser reconstruído, evitando perdas de informação causadas por fenômenos como o \textit{aliasing}, que ocorre quando a taxa de amostragem é inadequada.

Neste trabalho, analisamos os princípios da amostragem por três métodos: ideal, natural e \textit{flat-top} (topo plano), com base no teorema da amostragem, e implementamos simulações computacionais para ilustrar os efeitos práticos da amostragem e reconstrução de sinais digitais. A simulação, baseada na modulação por amplitude de pulso (PAM) de um sinal senoidal puro, incluiu a análise do espectro de frequências, a aplicação de métodos de filtragem e a reconstrução do sinal original. Utilizamos a Transformada de Fourier (FT) para análise espectral, o teorema de Nyquist para determinar a taxa de amostragem adequada, e filtros passa-baixa para a reconstrução do sinal. O documento descreve detalhadamente cada etapa do processo, os teoremas aplicados, os cálculos realizados e as ferramentas computacionais utilizadas, além de apresentar os resultados obtidos.


\section{Desenvolvimento}
% Aqui entra a pasta com o conteúdo de como a análise analítica foi realizada

$$
    n = \left[ \frac{\log{\frac{10^{\alpha_s/10}- 1}{10^{\alpha_p/10}-1}}}{2\log{\omega_s/\omega_p}} \right]
$$

Para um filtro num sinal com:
- Banda passante $\omega_p=2600 \, \text{Hz}$
- Banda stop $\omega_s = 3700\, \text{Hz}$
- Atenuação da banda passante $\alpha_p = 3dB$
- Atenuação da banda cortante $\alpha_s = 20dB$


\subsubsection{Executando o cálculo}
Incialmente, devemos realizar o cálculo da ordem do filtro:

\begin{align} \
    n = \left[ \frac{\log{\frac{10^{20/10}- 1}{10^{3/10}-1}}}{2\log{(2\pi \cdot3700/2\pi\cdot 2600)}} \right] \\
    n = \left[ \frac{\log{\frac{100- 1}{1.99526231497-1}}}{2\log{1.42307692308}} \right]                      \\
    n = \left[ \frac{\log{99.4712635161}}{0.30645675219} \right]                                              \\
    n = \left[ \frac{1.99769763453}{0.30645675219} \right]                                                    \\
    \\
    \therefore n = 6
\end{align}


Em seguida, obtemos o valor de $A_{max}$ para calcular a função de transferência do filtro.

$$
    H(j\omega) = \frac{1}{\sqrt{1+\epsilon^2 \left( \frac{\omega}{\omega_p} \right)^{2n}}}
$$

Como na fórmula, é preciso o valor de $\epsilon$, que é calculado da seguinte forma:

$$
    \epsilon = \sqrt{10^{A_{max}/10}-1}
$$

Considerando que o nosso $A_{max}=3\, \text{dB}$, temos que:

\begin{align} \
    \epsilon = \sqrt{10^{0.3}-1}    \\
    \epsilon = \sqrt{0.99526231497} \\
    \therefore \epsilon = 0.99762834511
\end{align}

Aplicando na fórmula da função de transferência:

\begin{align} \
    H(j\omega) = \frac{1}{\sqrt{1+0.9976^2 \left( \frac{\omega}{2\pi \cdot 2600} \right)^{2\cdot 6}}} \\
    H(j\omega) = \frac{1}{\sqrt{1+0.9953 \left( \frac{\omega}{2\pi \cdot 2600} \right)^{12}}}
\end{align} \\


\section{Resultados}
Nesta seção, apresentamos os resultados do nosso experimento de amostragem de sinais. A Figura~\ref{fig:input-signal}. mostra o sinal de entrada original, que serve como base para a aplicação das diferentes métodos de amostragem: ideal, natural e flat-top.

A análise deste sinal de entrada é crucial para entender a qualidade do sinal original e sua estrutura antes da modulação e amostragem. Observando a Figura~\ref{fig:input-signal}, podemos identificar as características principais do sinal que serão importantes para avaliar o impacto das técnicas de amostragem em etapas subsequentes.

Este sinal inicial será comparado com os sinais amostrados resultantes das técnicas aplicadas, permitindo uma análise detalhada das mudanças introduzidas durante o processo de amostragem e modulação.

\begin{figure}[H]
    \centering
    \includegraphics[width=1\linewidth]{03_results/octave_results/sinal_senoidal.png}
    \caption{Sinal senoidal de entrada}
    \label{fig:input-signal}
\end{figure}

\subsection{Ideal}

Na primeira parte da Figura~\ref{fig:ideal-pam}, observamos o sinal de trem de impulsos gerados, apresentados de forma discreto. O trem de impulsos é gerado para modular o sinal de entrada original e possibilitar a aplicação da amostragem com método ideal. Este trem é caracterizado teoricamente como conjunto infinito de impulsos unitários, de delta de dirac ($\delta$), espaçados de uma unidade, conforme a equação  \ref{tremimpulsos}. 

Após a geração do trem de impulsos, realizamos sua multiplicação junto ao sinal original, levando ao sinal como ilustrado na figura~\ref{fig:ideal-pam}, com \textit{aliasing} e na figura \ref{fig:ideal-pam-aliasing}, sem \textit{aliasing}. Esse processo resulta em um sinal modulado que reflete a amostragem do sinal original em intervalos discretos. 

Para realizar a execução das amostragens, foi necessário realizar a definição de uma frequência de amostragem, a qual é necessária para cada um dos métodos de amostragem apresentados conseguirem obter os valores corretos. Para a definição da frequência de amostragem, foi levada em consideração os valores retornados pelo código, dentre eles, o principal utilizado foi o valor da frequência máxima utilizada. Sendo $f_{max} = 910Hz; $
$F_s Nyquist = 2002Hz;$
$Aliasing F_s = 819Hz.$

\begin{figure}[H]
    \centering
    \includegraphics[width=0.8\linewidth]{03_results/octave_results/ideal_sampling.png}
    \caption{Saída sem Aliasing da amostragem Ideal}
    \label{fig:ideal-pam}
\end{figure}

\begin{figure}[H]
    \centering
    \includegraphics[width=0.8\linewidth]{03_results/octave_results/ideal_aliasing_sampling.png}
    \caption{Saída da amostragem Ideal - Frequencia de Amostragem = 819}
    \label{fig:ideal-pam-aliasing}
\end{figure}

\subsection{Natural}

Na Figura~\ref{fig:natural-pam}, observamos na primeira parte o trem de pulsos retangulares. Após gerado esse trem de pulsos, esse sinal foi multiplicado ao sinal original, gerando o sinal amostrado pelo método natural, como é possível observar na Figura~\ref{fig:natural-pam}. A amostragem onde acontece aliasing é possível observar na Figura~\ref{fig:natural-pam-aliasing}.

\begin{figure}[H]
    \centering
    \includegraphics[width=0.8\linewidth]{03_results/octave_results/natural_sampling.png}
    \caption{Saída sem Aliasing da amostragem Natural}
    \label{fig:natural-pam}
\end{figure}

\begin{figure}[H]
    \centering
    \includegraphics[width=0.8\linewidth]{03_results/octave_results/natural_aliasing_sampling.png}
    \caption{Saída da amostragem Natural - Frequencia de Amostragem = 819}
    \label{fig:natural-pam-aliasing}
\end{figure}

\subsection{Flat-Top}

Assim como na amostragem natural, é utilizado o trem de pulsos para realizar a modulação do sinal, como observado na Figura~\ref{fig:flattop-pam}. Mas, diferente da abordagem natural, no método de topo plano o trem de pulsos, quando modulado,  manterá o valor da amostragem até o próximo intervalo de amostragem $\tau$, conforme descrito no desenvolvimento das equações de \ref{eq:flattopFT} a  \ref{eq:flattop}. Os valores obtidos quando se tem \textit{aliasing} é possível observar na Figura~\ref{fig:flattop-pam-aliasing}.
\begin{figure}[H]
    \centering
    \includegraphics[width=0.8\linewidth]{03_results/octave_results/flattop_sampling.png}
    \caption{Saída sem Aliasing da amostragem Flat-Top}
    \label{fig:flattop-pam}
\end{figure}

\begin{figure}[H]
    \centering
    \includegraphics[width=0.8\linewidth]{03_results/octave_results/flattop_aliasing_sampling.png}
    \caption{Saída da amostragem Flat-Top - Frequência de Amostragem = 819}
    \label{fig:flattop-pam-aliasing}
\end{figure}

Nos três métodos de amostragem, quando a frequência de amostragem possui o valor abaixo do ideal, acontece o aliasing, resultando em um sinal reconstruído que não corresponde com o sinal original, resultando em perda de informações do sinal. Essa situação podendo ser vista na Figura~\ref{fig:ideal-pam-aliasing}, Figura~\ref{fig:natural-pam-aliasing} e Figura~\ref{fig:flattop-pam-aliasing}




\section{Discussões}
O desenvolvimento do produto enfrentou diversas problemáticas. Inicialmente, o projeto foi concebido para uma soma de senoides puras com uma faixa de frequência delimitada, o que gerou previsibilidade nos resultados. Contudo, ao utilizar um sinal de teste fornecido (com senoides e cossenoides, mas uma faixa de frequência menos previsível), observou-se um comportamento diferente no PAM, exigindo ajustes na frequência de corte e no ciclo de trabalho (\textit{duty cycle}) para adequação do PAM, além de refinamentos na reconstrução do sinal.

O sinal de teste impediu o uso da mesma frequência de amostragem de 8KHz, obrigando a adoção de frequências mais altas. Isso permitiu um desempenho mais adequado do PAM e maior proximidade entre o sinal reconstruído e o original.

Outra questão foi o uso de filtros passa-baixa, que tiveram papel importante tanto na modulação quanto na demodulação, garantindo as condições do Teorema de Nyquist-Shannon. Um filtro \textit{Butterworth} foi utilizado para evitar aliasing, embora tenha surgido dificuldade na implementação inicial, pois o filtro \textit{anti-aliasing} não foi considerado. Em alguns momentos, a reconstrução do sinal apresentou melhores resultados sem o filtro, sugerindo possíveis erros na implementação ou inadequação da frequência de corte.

O filtro passa-baixa também foi aplicado na demodulação, com desafios semelhantes, mas menos acentuados. A escolha de uma ordem diferente para o \textit{Butterworth} no \textit{anti-aliasing} (ordem 4) e na demodulação (ordem 6) pode ter contribuído para esses desafios. Além disso, alguns testes exigiram amplificação do sinal devido a perdas, o que foi solucionado com ajustes nas frequências e filtros.

\section{Conclusões}
Este experimento explorou conceitos fundamentais, como o Teorema da Amostragem e a Transformada Rápida de Fourier, essenciais para a modulação e conversão analógico-digital (A/D). As simulações destacaram a importância de escolher uma frequência de amostragem adequada, utilizando filtros \textit{anti-aliasing} para evitar o \textit{aliasing}, garantindo a integridade do sinal e sua reconstrução fiel.

As análises gráficas e matemáticas realizadas reforçaram o papel crucial do Teorema de Nyquist, demonstrando que uma amostragem precisa e a aplicação de filtros prévios são essenciais para evitar distorções no processo de reconstrução. A correta filtragem permitiu a preservação das características originais do sinal.

Conclui-se que a implementação de filtros \textit{anti-aliasing} e a escolha adequada da taxa de amostragem foram determinantes para garantir a fidelidade do sinal reconstruído, destacando a importância desses elementos no desenvolvimento de sistemas de processamento de sinais digitais robustos e eficientes.

\begin{thebibliography}{00}
\bibitem{b1} SANCA, A.S. TEC430 - PROCESSAMENTO DIGITAL DE SINAIS : Notas de Aula. Universidade Estadual De Feira De Santana - Departamento De Tecnologia (DTEC). 2024.

\bibitem{b2} Alan V. Oppenheim, Alan S. Willsky, and S. Hamid Nawab. Sinais e Sistemas. Prentice Hall, Sao Paulo, SP, Brasil, 3ª edição, 2012.
\bibitem{b3} L W. Couch. Digital and Analog Communication Systems. Prentice Hall, University of Florida (Electrical and Computer Engineering), New Jersey, 2ª edição, 2007.
\end{thebibliography}

\end{document}
