A análise espectral de sinais é amplamente utilizada em áreas como telecomunicações, eletrônica e processamento de áudio e imagens. Essa técnica decompõe sinais em suas componentes de frequência, permitindo identificar padrões repetitivos, ruídos e distorções, essenciais para entender sinais temporais e imagens.

Neste trabalho, exploramos a análise espectral de sinais digitais de audio e imagem, utilizando ferramentas computacionais para investigar o comportamento espectral e aplicar técnicas de filtragem no domínio da frequência.

Aplicamos a Transformada de Fourier 2D para analisar imagens no domínio da frequência e a Transformada de Fourier unidimensional para áudios ruidosos. Utilizamos filtros clássicos, como \textit{Butterworth}, \textit{Chebyshev} e \textit{Elíptico}, para remover ruídos e melhorar a qualidade.

Comparamos os efeitos de cada filtro na preservação de detalhes e na eficiência da redução de ruídos, destacando suas principais vantagens e limitações.
