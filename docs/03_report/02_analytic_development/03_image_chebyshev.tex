Para que pudéssemos aplicar corretamente o filtro, foi necessário o cálculo da ordem do mesmo.

Foram definidos os seguintes parâmetros para o filtro \textit{Chebyshev} rejeita-faixa:

\begin{enumerate}
    \item Banda passante $\omega_p= [98, 102] \, \text{Hz}$
    \item Banda de rejeição $\omega_s = [99, 101]\, \text{Hz}$
    \item Atenuação da banda passante $\alpha_p = 0.1dB$
    \item Atenuação da banda de rejeição $\alpha_s = 60dB$
\end{enumerate}


\subsubsection*{Executando o cálculo}
Incialmente, devemos realizar o cálculo da ordem do filtro, que segue a fórmula:

$$
    n = \text{ceil} \left[ \frac{\cosh^{-1}{\sqrt{\frac{10^{\alpha_s/10}- 1}{10^{\alpha_p/10}-1}}}}{\cosh^{-1}{\omega_s/\omega_p}} \right]
$$


Como as bandas de passante e rejeição são dadas como intervalos, devemos usar a frequência central de cada intervalo.

\subsubsection*{Calculando o denominador}

Quando as frequências são fornecidas em Hertz (Hz), elas precisam ser normalizadas em relação à frequência de Nyquist. A frequência de Nyquist é dada por $f_{\text{Nyquist}} = \frac{f_s}{2}$, onde $f_s$ é a taxa de amostragem.

Neste caso, a taxa de amostragem $f_s$ é 256 Hz. Então, as frequências passante ($\omega_p$) e rejeição ($\omega_s$) precisam ser ajustadas dividindo cada valor pela frequência de Nyquist.

A fórmula para normalizar as frequências $\omega_p$ e $\omega_s$ é:

\begin{align*} \
    \omega_p = \frac{\text{Freqüência de passante}}{128}                       \\
    \omega_s = \frac{\text{Freqüência de rejeição}}{128}                       \\
    \\
    \text{Substituindo os valores}                                             \\
    \\
    \omega_p = \left[\frac{98}{128}, \frac{102}{128}\right] = [0.7656, 0.7969] \\
    \omega_s = \left[\frac{99}{128}, \frac{101}{128}\right] = [0.7734, 0.7891]
\end{align*}

Em seguida, aplicamos a transformação tangente para essas frequências normalizadas, utilizando a seguinte fórmula:

\begin{align*} \
    \text{passb} = \tan\left(\frac{\pi \cdot \omega_p}{2}\right)         \\
    \text{stopb} = \tan\left(\frac{\pi \cdot \omega_s}{2}\right)         \\
    \\
    \text{Aplicando a transformação tangente}                            \\
    \\
    \text{passb} = \tan\left(\frac{\pi \cdot [0.7656, 0.7969]}{2}\right) \\ \quad \Rightarrow \quad \text{passb} = [2.592099, 3.027442] \\
    \text{stopb} = \tan\left(\frac{\pi \cdot [0.7734, 0.7891]}{2}\right) \\ \quad \Rightarrow \quad \text{stopb} = [2.689781, 2.907342]
\end{align*}


Com as frequências passante ($\text{passb}$) e rejeição ($\text{stopb}$) transformadas, podemos calcular o valor de $\text{freqRatio}$ usando a fórmula:

$$
    \text{freqRatio} = \text{min}_{\text{absoluto}} \left(\frac{\text{stopb} \cdot (\text{passb}[1] - \text{passb}[2])}{\text{stopb}^2 - \text{passb}[1] \cdot \text{passb}[2]} \right)
$$

Substituindo os valores:

\begin{align*} \
    \text{freqRatio} = \frac{\{2.689781, 2.907342\} \cdot (2.592099 - 3.027442)}{\{2.689781, 2.907342\}^2 - 2.592099 \cdot 3.027442} \\
    \text{freqRatio} = \frac{\{2.689781, 2.907342\} \cdot (-0.4346)}{\{7.2392, 8.4526\} - 7.8541}                      \\
    \\
    \begin{array}{cc}
        \text{freqRatio} = \frac{-1.1700}{-0.6149} & \text{freqRatio} = \frac{-1.2635}{0.5985} \\
        \text{freqRatio} = 1.9027                  & \text{freqRatio} = -2.1111
    \end{array}
    \\
    \therefore \text{freqRation} = 1.9027
\end{align*}

Este é o valor de $\text{freqRatio}$ calculado a partir das frequências normalizadas e transformadas.

Agora, vamos substituir os valores conhecidos na fórmula para calcular a ordem do filtro:

\begin{align*} \
    n = \text{ceil} \left[ \frac{\cosh^{-1}{\sqrt{\frac{10^{60/10}- 1}{10^{1/10}-1}}}}{\cosh^{-1}{freqRatio}} \right] \\
    n = \text{ceil} \left[ \frac{\cosh^{-1}{\sqrt{\frac{999999}{0.2589}}}}{\cosh^{-1}{freqRatio}} \right]            \\
    n = \text{ceil} \left[ \frac{\cosh^{-1}{\sqrt{3862491.30939}}}{\cosh^{-1}{1.9027}} \right]                           \\
    n = \text{ceil} \left[ \frac{\cosh^{-1}{1965.3222}}{\cosh^{-1}{1.9027}} \right]                                  \\n = \text{ceil} \left[ \frac{8.27656}{1.2589} \right]                                     \\
    n = \text{ceil} [6.5744]                                                                                      \\
    \therefore n = 7
\end{align*}
