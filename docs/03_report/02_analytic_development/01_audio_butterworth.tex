Para que pudéssemos aplicar corretamente o filtro, foi necessário o cálculo da ordem do mesmo, bem como a atenuação da banda passante ($\epsilon$).

Foram definidos os seguintes parâmetros para o filtro \textit{Butterworth} passa-baixa:

\begin{enumerate}
    \item Banda passante $\omega_p=1000 \, \text{Hz}$
    \item Banda de rejeição $\omega_s = 2000\, \text{Hz}$
    \item Atenuação da banda passante $\alpha_p = 1dB$
    \item Atenuação da banda de rejeição $\alpha_s = 20dB$
\end{enumerate}


\subsubsection*{Executando o cálculo}
Incialmente, devemos realizar o cálculo da ordem do filtro, que segue a fórmula:

\begin{align} \
    n = \left[ \frac{\log{\frac{10^{\alpha_s/10}- 1}{10^{\alpha_p/10}-1}}}{2\log{\omega_s/\omega_p}} \right] \\
    \label{eq:butterworth_n_formula}
\end{align}

Substituindo-se os valores, na fórmula, temos que:
\begin{align*} \
    n = \left[ \frac{\log{\frac{10^{20/10}- 1}{10^{1/10}-1}}}{2\log{(2\pi \cdot2000/2\pi\cdot 1000)}} \right] \\
    n = \left[ \frac{\log{\frac{100- 1}{1.25892541179-1}}}{2\log{2}} \right]                                  \\
    n = \left[ \frac{\log{382.349493298}}{0.60205999132} \right]                                              \\
    n = \left[ \frac{2.58246051898}{0.60205999132} \right]                                                    \\
    n = 4.28937407603                                                                                         \\
    \\
    \therefore n = 5
\end{align*}


Em seguida, obtemos o valor de $A_{max}$ para calcular a função de transferência do filtro.

$$
    H(j\omega) = \frac{1}{\sqrt{1+\epsilon^2 \left( \frac{\omega}{\omega_p} \right)^{2n}}}
$$

Como na fórmula, é preciso o valor de $\epsilon$, que é calculado da seguinte forma:

\begin{align} \
    \epsilon = \sqrt{10^{A_{max}/10}-1} \
    \label{eq:butterworth_epsilon_calculation} \
\end{align}

Considerando que o nosso $A_{max}=1\, \text{dB}$, temos que:

\begin{align*} \
    \epsilon = \sqrt{10^{1/10}-1}   \\
    \epsilon = \sqrt{0.25892541179} \\
    \therefore \epsilon = 0.509
\end{align*}

Agora, aplicando o valor de $\epsilon$ e $n$ na fórmula da função de transferência, temos:

\begin{align} \
    H(j\omega) = \frac{1}{\sqrt{1 + 0.509^2 \left( \frac{\omega}{1000} \right)^{10}}} \\
    H(j\omega) = \frac{1}{\sqrt{1 + 0.259 \left( \frac{\omega}{1000} \right)^{10}}}
\end{align}
