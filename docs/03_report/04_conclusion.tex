Neste trabalho, realizamos a análise espectral de sinais e imagens digitais, utilizando ferramentas computacionais avançadas como a Transformada Discreta de Fourier (DTFT) e sua extensão bidimensional (2D-FFT).

Além disso, a análise espectral foi estendida ao domínio das imagens, onde a Transformada de Fourier 2D permitiu a conversão das imagens do espaço para a frequência. Isso possibilitou a aplicação de filtros como o Butterworth e Chebyshev. Os filtros foram comparados em termos de suas respostas de frequência e impacto visual nas imagens filtradas.

O filtro Butterworth mostrou-se eficaz na remoção de ruído de alta frequência sem causar distorções perceptíveis, sendo uma excelente escolha para preservação de detalhes importantes nas imagens. O filtro Chebyshev, mais agressivo, proporcionou uma remoção eficiente de ruídos, mas com o custo de introduzir distorções em áreas suaves das imagens.

No que diz respeito a filtragem no áudio, notou-se também a grande eficácia do filtro \textit{Butterworth}, o qual eficientemente removeu as altas frequências enquanto preservava o áudio original. Enquanto o filtro Elíptico, por sua vez, apresentou uma filtragem eficaz com atenuação mais agressiva, mas com a desvantagem das oscilações na resposta de magnitude, que podem resultar em distorções perceptíveis.

Por fim, a análise espectral e a aplicação dos filtros demonstraram ser ferramentas poderosas no processamento de sinais e imagens, permitindo não apenas a identificação e a remoção de ruídos, mas também a preservação de características essenciais do sinal ou imagem. A escolha adequada do filtro, em conjunto com a análise espectral, é crucial para otimizar a qualidade do sinal ou imagem processada.
