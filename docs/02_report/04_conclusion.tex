Neste estudo, foi possível observar a importância da escolha da frequência de amostragem e do número de amostras para uma análise espectral precisa de sinais discretos. Ao variar a quantidade de amostras e aplicar janelas ao sinal, identificamos o impacto direto desses parâmetros na resolução espectral e na precisão do espectro obtido. A aplicação da janela retangular mostrou-se eficaz para minimizar o vazamento espectral, embora introduza um alargamento dos picos espectrais, reduzindo a resolução.

Através do uso da FFT para aproximar a DTFT, verificou-se que a técnica é eficiente para analisar sinais com diferentes frequências de amostragem, desde que o critério de Nyquist seja respeitado. Notamos que, mesmo com um número reduzido de amostras, as frequências principais do sinal podem ser identificadas, permitindo a reconstrução do sinal original. No entanto, para frequências muito próximas, uma maior quantidade de amostras é essencial para distinguir adequadamente as componentes.

Este trabalho reforça a importância de ajustar o número de amostras e o tipo de janela para obter uma análise espectral confiável, com sugestões para experimentos futuros incluindo o uso de outras janelas (como Hamming ou Hann) para análises do comportamento espectral.
