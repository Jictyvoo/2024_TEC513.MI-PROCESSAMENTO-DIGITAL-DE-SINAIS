A análise espectral de sinais é uma técnica fundamental em processamento de sinais, amplamente aplicada em áreas como telecomunicações, eletrônica e processamento de dados. Ela permite decompor um sinal no domínio do tempo em suas componentes de frequência, revelando as frequências presentes e suas respectivas amplitudes, o que facilita a identificação de características essenciais do sinal, como componentes harmônicas, ruídos e distorções.

Neste trabalho, foi realizada a análise espectral de sinais amostrados a partir do Octave. Para atingir esse objetivo, necessitou-se utilizar o método prático da Transformada Discreta de Fourier no Tempo (DTFT). Visando testar e obter resultados referentes à análise espectral, foi utilizado um gerador de funções, enquanto que foi utilizado um microcontrolador Arduino ATMega2560 para realizar o processo de amostragem do sinal. Esses dados foram transferidos para o Octave por meio de uma porta serial, e a partir daí, aplicada a função FFT a esse sinal amostrado, implementando de maneira prática a DTFT para observar o comportamento espectral do sinal.

Para minimizar o vazamento espectral e melhorar a definição do espectro, o sinal foi janelado antes da aplicação da transformada. A quantidade de amostras foi também um parâmetro importante, pois afeta diretamente a resolução espectral e a precisão dos resultados no domínio da frequência. Com essa abordagem, buscamos identificar e interpretar as componentes de frequência predominantes, visando caracterizar a estrutura do sinal e seu comportamento espectral em diferentes taxas de amostragem.
