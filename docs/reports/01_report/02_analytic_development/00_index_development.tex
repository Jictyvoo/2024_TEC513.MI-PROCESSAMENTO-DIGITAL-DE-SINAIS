O desenvolvimento da solução para o problema apresentado ocorreu a partir de quatro etapas principais: 1. aplicação de um filtro \textit{anti-aliasing}, 2. amostragem por Modulação por Amplitude de Pulso (PAM), 3. obtenção do espectro do sinal amostrado e 4. reconstrução do sinal no domínio do tempo.

Na entrada do sistema, temos uma soma de sinais senoidais puros que pode ser expressa na forma:

\begin{equation}
    x(t) = a \cdot \sin(2\pi f_1 \cdot t) + b \cdot \sin(2\pi f_2 \cdot t)
\end{equation}

Sinais contínuos, como este, geralmente não são limitados em banda, o que pode resultar no fenômeno de \textit{aliasing}. Para evitá-lo, aplicaremos o Teorema da Amostragem através de um filtro passa-baixa \textit{Butterworth}, que funciona como um filtro \textit{anti-aliasing}. Os filtros dessa classe possuem diversas propriedades, como a resposta de magnitude na banda de passagem, tornando-os atraentes para implementações práticas \cite{b2}.

Um filtro passa-baixa \textit{Butterworth} de ordem $N$ tem uma resposta em frequência cujo quadrado da magnitude é dado por:

\begin{equation}
    |B(jw)|^2 = \frac{1}{1 + \left( \frac{jw}{jw_c} \right)^{2N}} 
\end{equation}

Durante as análises matemáticas, utilizamos frequentemente as propriedades de seno e cosseno. Uma identidade fundamental é:

\begin{equation}
    \cos(\theta) \pm j\sin(\theta) = e^{\pm j\theta}
\end{equation}

A partir dessa equação, proveniente da análise do círculo trigonométrico, podemos deduzir duas expressões ao somar e subtrair $e^{-j\theta}$ de $e^{j\theta}$:

\begin{align}
    e^{j\theta} - e^{-j\theta} &= 2j\sin(\theta) \\
    e^{j\theta} + e^{-j\theta} &= 2\cos(\theta)
\end{align}

Com esses fundamentos do círculo trigonométrico, seguimos para o desenvolvimento dos cálculos necessários para os métodos de amostragem.

\subsection{Amostragem ideal}
Ao modular um sinal qualquer, $x(t)$, com um trem de impulsos, $p(t)$ podemos obter o valor instantâneo da função de acordo com o período do trem de impulsos. Dessa forma, a amostragem ideal se dá como um trem de impulsos ponderados pela amplitude de uma função qualquer. E pode ser calculado no domínio do tempo como:
\begin{equation}
    i(t) = x(t) \cdot p(t)
\end{equation}
Considerando que o trem de impulsos é representado como:

\begin{equation}
p(t) = \sum_{n=-\infty}^{\infty} \delta(t - n\tau)
    \label{tremimpulsos}
    \end{equation}


onde $\delta(t)$ é a função delta de Dirac e $\tau$ é o período do trem de impulsos.

Para calcular a amostragem no domínio da frequência, precisamos primeiramente definir a transformada de Fourier $P(j\omega)$ desse trem de impulsos. Utilizando a propriedade de deslocamento no tempo da transformada de Fourier e a transformada da função delta de Dirac. Sabemos que:

$$
\mathcal{F}\{\delta(t - n\tau)\} = e^{-j\omega n\tau}
$$

Por definição, a transformada de um impulso deslocado se dá por
$$
\int_{-\infty}^{\infty} \delta(t - \tau) \cdot e^{-j\omega (t)} \, dt
$$

Resolvendo a integral através do método de substituição, considerando $u = t - \tau$ e $du = dt$:
$$
\int_{-\infty}^{\infty} \delta(u) \cdot e^{-j\omega (u + \tau)} \, du
= \int_{-\infty}^{\infty} \delta(u) \cdot e^{-j\omega u} \cdot e^{-j\omega \tau} \, du
$$
$$
= e^{-j\omega \tau} \cdot \int_{-\infty}^{\infty} \delta(u) \cdot e^{-j\omega u} \, du
$$

A função de impulso é definida como:
\begin{equation}
    \begin{cases}
    \int_{-\infty}^{\infty} \delta(t) \, dt &= 1 \\
    0, t \neq 0
    \end{cases}
\end{equation}
Dessa forma, o único momento em que a função $\delta$ assume um valor diferente de zero é quando $u$ estiver se aproximando de 0. Nessa condição, a exponencial que se encontra dentro da integral assume valor 1, portanto, a transformada de um impulso deslocado é:
\begin{equation}
\int_{-\infty}^{\infty} \delta(t - \tau) \cdot e^{-j\omega (t)} \, dt = e^{-j\omega \tau}
\end{equation}

Assim, a transformada de Fourier do trem de impulsos deslocados, $p(t)$, é dada por:

$$
P(j\omega) = \int_{-\infty}^{\infty} x(t) \, e^{-j\omega t} \, dt
$$
$$
= \int_{-\infty}^{\infty} \sum_{n=-\infty}^{\infty} \delta(t - n\tau) \, e^{-j\omega t} \, dt
$$

Podemos usar o princípio de integrais para fazer o somatório das integrais (invertendo a ordem de soma e integral):

$$
P(j\omega) = \sum_{n=-\infty}^{\infty} \int_{-\infty}^{\infty} \delta(t - n\tau) \, e^{-j\omega t} \, dt \\
$$
\begin{equation}
\therefore
P(j\omega) = \sum_{n=-\infty}^{\infty} e^{-j\omega n\tau}
\end{equation}

\subsubsection{Aplicação na Transformada de Fourier do Trem de Impulsos}

No caso do trem de impulsos, o sinal no tempo $p(t)$ é uma soma infinita de funções delta de Dirac deslocadas no tempo:

De acordo com o que já calculamos, a integral para cada $n$ pode ser representada como:
$$
P(j\omega) = \sum_{n=-\infty}^{\infty} e^{-j\omega n\tau}
$$
Esta soma é uma série de exponenciais complexas. O princípio da ortogonalidade é aplicado aqui para reconhecer que essa soma se comporta como um trem de impulsos no domínio da frequência. Especificamente, a soma infinita de exponenciais complexas, quando avaliada em frequências específicas, resulta em uma série de deltas de Dirac.

Primeiro, reconhecemos que a soma acima é uma série geométrica infinita se pensarmos em $e^{-j\omega\tau}$ como a razão comum:
$$
S(\omega) = \sum_{n=-\infty}^{\infty} \left(e^{-j\omega \tau}\right)^n
$$
Para formalizar, expandimos a soma $S(\omega)$ em uma série de Fourier. Considerando que $S(\omega)$ é periódica com período $\frac{2\pi}{\tau}$, podemos escrevê-la como:

$$
S(\omega) = \sum_{k=-\infty}^{\infty} c_k \delta\left(\omega - k\frac{2\pi}{\tau}\right)
$$

Aqui, $c_k$ são os coeficientes da série de Fourier, que no caso de um trem de impulsos, assumem um valor constante.

\subsubsection{Determinação dos Coeficientes}

Os coeficientes $c_k$ podem ser determinados integrando $S(\omega)$ em um período:

$$
c_k = \frac{1}{\tau} \int_{-\frac{\pi}{\tau}}^{\frac{\pi}{\tau}} e^{-j\omega n\tau} d\omega = \frac{2\pi}{\tau}
$$

Esse valor é devido à ortogonalidade das exponenciais complexas e ao fato de estarmos somando sobre um trem de impulsos no tempo.

$$
\sum_{n=-\infty}^{\infty} e^{-j\omega n\tau} = \frac{2\pi}{\tau} \sum_{k=-\infty}^{\infty} \delta\left(\omega - k\frac{2\pi}{\tau}\right)
$$

Portanto, o valor final é:
\begin{equation}
    I(j\omega) = \frac{2\pi}{\tau} \sum_{k=-\infty}^{\infty} \delta\left(\omega - k\frac{2\pi}{\tau}\right)    
\end{equation}
Este resultado mostra que a transformada de Fourier de um trem de impulsos com período $\tau$ é um trem de impulsos no domínio da frequência, com espaçamento $\frac{2\pi}{\tau}$.


\subsection{Amostragem natural}
Esse método de amostragem é resultado da modulação de um trem de pulsos, $c(t)$, por uma função qualquer, $x(t)$. Similar a amostragem ideal, a amostragem natural se dá matematicamente como:
\begin{align}
    n(t) = x(t) \cdot c(t)
\end{align}

% $$
% Transformada do trem de pulsos
% C(j\omega) = \frac{2\pi\Delta}{T} \sum_{-\infty}^{\infty} sinc(k\frac{\Delta}{T}) \cdot X(j\omega-jk\omega_0)
% $$

\begin{equation}
    p(t) = \sum_{n=-\infty}^{\infty} \delta(t - n\tau)    
\end{equation}
    \begin{equation}
    P(j\omega) = \frac{2\pi}{\tau} \sum_{n=-\infty}^{\infty} \delta(\omega - n\frac{2\pi}{\tau})    
    \end{equation}
    \begin{equation}
    p(t) \overset{\text{FT}}{\longleftrightarrow} P(j\omega)     
    \end{equation}
    \begin{equation}
    \delta(t) \overset{\text{FT}}{\longleftrightarrow} 1        
    \end{equation}
    \begin{equation}
    \delta(t - n\tau) \overset{\text{FT}}{\longleftrightarrow} e^{-j\omega n \tau}        
    \end{equation}
    \begin{equation}
        S(j\omega) = \frac{2\pi}{\tau} \sum_{n=-\infty}^{\infty} \delta(\omega - n\omega_{0}) a_{k} \cdot e^{j\omega_{0}t} \overset{\text{FT}}{\longleftrightarrow} 2\pi a_{k}\delta(\omega-\omega_{0})
    \end{equation}
    \begin{equation}       
    s(t)\overset{\text{FT}}{\longleftrightarrow}S(j\omega)
    \end{equation}
    \begin{equation}
    \sum_{n=-\infty}^{\infty} a_{n}e^{j\omega_{0}nt} \overset{\text{FT}}{\longleftrightarrow} \sum_{n=-\infty}^{\infty} 2\pi a_{n}\delta(\omega-n\omega_{0})
            \end{equation}
            \begin{equation}
    \delta(t-n\tau) \overset{\text{FT}}{\longleftrightarrow} e^{-j\omega n\tau}              
            \end{equation}
\begin{equation}
    \phi(t-n\tau) \overset{\text{FT}}{\longleftrightarrow} a_{k}e^{-j\omega n\tau} 
\end{equation}

Para continuar a realizar o desenvolvimento da equação, podemos notar que a única mudança para com relação ao resultado da transformada para um trem de impulsos, é o valor proveniente de $a_k$, o qual pode ser definido como a seguir:

\begin{align}
    a_k = \frac{1}{\tau} \int_{-\frac{\Delta}{2}}^{\frac{\Delta}{2}} 1 \cdot e^{-jk \frac{2\pi}{\tau}t}dt \\
    = \frac{1}{\tau} (\frac{e^{-jk\frac{2\pi}{\tau}}}{-jk\frac{2\pi}{\tau}}) \bigg\rvert_{-\Delta/2}^{\Delta/2} \\
    = \frac{-1}{jk2\pi} e^{-jk\frac{2\pi}{\tau}\frac{\Delta}{2}} - e^{jk\frac{2\pi}{\tau}\frac{\Delta}{2}} \\
    = \frac{1}{k\pi}\frac{e^{jk\frac{\pi}{\tau}\Delta}-e^{-jk\frac{\pi}{\tau}\Delta}}{2j} \\
    = \frac{1}{k\pi} sen(k\pi\frac{\Delta}{\tau})\frac{\Delta/\tau}{\Delta/\tau} \\
    a_{k} = \frac{\Delta}{\tau} sinc(k\frac{\Delta}{\tau})
\end{align}


\subsection{Amostragem de topo plano (flat-top)}
Para esse método, a cada instante de tempo o amostrador assume o valor atual da função $x(t)$, esse valor é mantido até o próximo instante em que uma nova amostra é tomada.Para isso, podemos realizar a convolução da amostragem ideal $i(t)$ com um pulso retangular $r(t)$, obtendo assim um trem de pulsos retangulares ponderados pela amostragem ideal. 

\begin{align}
    f(t) = i(t) * r(t) \overset{\text{FT}} {\longleftrightarrow} F(j\omega) = I(j\omega) \cdot R(j\omega)
    \label{eq:flattopFT}
\end{align}


A partir da definição do pulso retangular simétrico, temos o desenvolvimento de sua transformada.

\begin{align}
r(t) = 
\begin{cases} 
1, & \text{se } \frac{-\Delta}{2} \leq t \leq \frac{\Delta}{2}, \\ 
0, & \text{caso contrário}
\end{cases}
\end{align}

\begin{equation}
R(j\omega) = \int_{-\infty}^{\infty} r_2(t) e^{-j\omega t}dt
\end{equation}
\begin{equation}
= \int_{-\frac{\Delta}{2}}^{z\frac{\Delta}{2}} e^{-j\omega t}dt
\end{equation}
\begin{equation}
 = -\frac{1}{j\omega}\cdot(e^{-j\omega t} \bigg\rvert_{-\Delta/2}^{\Delta/2})
\end{equation}
\begin{equation}
= -\frac{1}{j\omega}\left(e^{-\frac{j\omega\Delta}{2}}-e^{\frac{j\omega\Delta}{2}}\right)
\end{equation}

A partir da relação de exponencial complexa com funções senoidais, dada em (1), podemos simplificar como:
\begin{equation}
R(j\omega) = \frac{1}{j\omega} \cdot 2j\sin{\left(\frac{\omega\Delta}{2}\right)}
\end{equation}
\begin{equation}
R(j\omega) = \frac{2}{\omega} \sin{\left(\frac{\omega\Delta}{2}\right)} \cdot \frac{\Delta/2}{\Delta/2}
\end{equation}
\begin{align}
    R(j\omega) = \Delta \cdot \operatorname{sinc}\left(\frac{\omega\Delta}{2}\right)
\end{align}

Assim sendo, temos:
\begin{equation}
 F(j\omega) = I(j\omega) \cdot R(j\omega)
\end{equation}
\begin{equation}
 = \frac{2\pi}{\tau}\sum_{k=-\infty}^{\infty}X(j(\omega-k\frac{2\pi}{\tau})) \cdot \Delta sinc\left(\frac{\omega\Delta}{2}\right)
\end{equation}
\begin{align}
F(j\omega) = \frac{2\pi}{\tau} \Delta sinc\left(\frac{\omega\Delta}{2}\right) \sum_{k=-\infty}^{\infty}X\left(j\left(\omega-k\frac{2\pi}{\tau}\right)\right)
\label{eq:flattop}
\end{align}

