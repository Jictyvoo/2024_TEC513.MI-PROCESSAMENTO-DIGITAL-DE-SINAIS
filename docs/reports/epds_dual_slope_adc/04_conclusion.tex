O conversor A/D de rampa dupla apresenta características que o tornam uma opção viável para aplicações específicas. Ele é projetado com componentes de baixa complexidade, o que resulta em um custo reduzido, mas ainda assim oferece boa resolução. Além disso, é pouco suscetível a ruídos e é insensível à mudanças mínimas dos valores de R e C por fatores externos, o que aumenta sua robustez.

Uma de suas principais limitações, no entanto, é o tempo de conversão lenta, visto que é necessário o período de integração e desintegração inteiros para uma única conversão. Outra limitação é a necessidade de um \textit{clock} cada vez maior no contador do circuito de acordo com a frequência da entrada, tornando sua implementação inviável para sinais que possuem uma frequência muito alta, pois resultaria em \textit{clocks} muito custosos ou impraticáveis. Por isso, ele é mais indicado para sinais bastante lentos e é amplamente utilizado em instrumentos de medição, como multímetros digitais de uso comercial, onde sua simplicidade, custo-benefício, resistência a ruídos e resolução são vantagens consideráveis.
