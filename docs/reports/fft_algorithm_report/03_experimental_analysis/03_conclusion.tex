Conclui-se que a quantidade de amostras desempenha um papel crucial na análise discreta do sinal, uma vez que influencia diretamente a resolução espectral e a precisão na identificação das componentes de frequência.

O aumento no número de amostras resulta em uma representação mais detalhada do sinal, permitindo uma discriminação mais precisa das frequências dominantes e aprimorando a qualidade da análise espectral.
