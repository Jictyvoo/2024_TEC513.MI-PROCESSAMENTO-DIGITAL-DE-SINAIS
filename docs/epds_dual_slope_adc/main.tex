
\documentclass[conference]{IEEEtran}
\IEEEoverridecommandlockouts
% The preceding line is only needed to identify funding in the first footnote. If that is unneeded, please comment it out.
\usepackage{cite}
\usepackage{amsmath,amssymb,amsfonts}
\usepackage{algorithmic}
\usepackage{graphicx}
\usepackage{textcomp}
\usepackage{xcolor}
\usepackage{float}
\usepackage{listings}
\usepackage{svg}
\usepackage{hyperref}

\usepackage{xcolor}

\definecolor{codegreen}{rgb}{0,0.6,0}
\definecolor{codegray}{rgb}{0.5,0.5,0.5}
\definecolor{codepurple}{rgb}{0.58,0,0.82}
\definecolor{backcolour}{rgb}{0.95,0.95,0.92}
\DeclareUnicodeCharacter{03BC}{\ensuremath{\mu}}


\lstdefinestyle{mystyle}{
    backgroundcolor=\color{backcolour},   
    commentstyle=\color{codegreen},
    keywordstyle=\color{magenta},
    numberstyle=\tiny\color{codegray},
    stringstyle=\color{codepurple},
    basicstyle=\ttfamily\footnotesize,
    breakatwhitespace=false,         
    breaklines=true,                 
    captionpos=b,                    
    keepspaces=true,                 
    numbers=left,                    
    numbersep=5pt,                  
    showspaces=false,                
    showstringspaces=false,
    showtabs=false,                  
    tabsize=2
}

\lstset{style=mystyle}

\def\BibTeX{{\rm B\kern-.05em{\sc i\kern-.025em b}\kern-.08em
    T\kern-.1667em\lower.7ex\hbox{E}\kern-.125emX}}

\begin{document}

\makeatletter
\newcommand{\linebreakand}{%
    \end{@IEEEauthorhalign}
    \hfill\mbox{}\par
    \mbox{}\hfill\begin{@IEEEauthorhalign}
}
\makeatother

\title{Problema 1: Análise Matemática Comparativa da Amostragem de Sinais Analógicos\\
}

\author{
    \IEEEauthorblockN{1\textsuperscript{st} João Victor Oliveira Couto}
    \IEEEauthorblockA{\textit{Estudante de Eng. de Computação} \\
        \textit{Universidade Estadual de Feira de Santana}\\
        Feira de Santana, Bahia, Brasil \\
        contact@jityvoo.dev}
    \and
    \IEEEauthorblockN{2\textsuperscript{nd} Marcus Aldrey}
    \IEEEauthorblockA{\textit{Estudante de Eng. de Computação} \\
        \textit{Universidade Estadual de Feira de Santana}\\
        Feira de Santana, Bahia, Brasil \\
        marcusaldrey@gmail.com}
    \IEEEauthorblockN{}
    \IEEEauthorblockA{}
}

\maketitle

\begin{abstract}
    Este trabalho apresenta a simulação e análise de um ADC Dual-Slope de 6 bits usando o simulador Falstad. O sistema converte um sinal de entrada senoidal de 1 Hz em uma saída digital quantizada. A simulação demonstra o processo de integração e quantização, destacando os 64 níveis discretos determinados pela resolução de 6 bits. Os resultados validam a funcionalidade do ADC, enfatizando sua adequação para aplicações de baixo custo e baixa precisão.
\end{abstract}

\begin{IEEEkeywords}
    ADC Rampa Dupla; Quantização de sinal; Resolução de 6 bits
\end{IEEEkeywords}

\section{Introdução}
A conversão de sinais analógicos em processamento digital de sinais é fundamental em telecomunicações, redes de computadores, sistemas inteligentes, aplicações industriais, militares e governamentais. Isso se deve ao fato de que muitos sinais naturais, como grandezas físicas, são analógicos ou contínuos, e para que possam ser processados em dispositivos eletrônicos e computacionais, precisam ser convertidos para uma representação discreta. 

\begin{figure}[H]
    \centering
    \includegraphics[width=1\linewidth]{image.png}
    \caption{Processamento a tempo discreto de sinais a tempo contínuo. Fonte: SANCA (2024)}
    \label{fig:enter-label}
\end{figure}
O processo de conversão analógico-digital (A/D) envolve três etapas principais: amostragem, quantização e codificação, sendo a amostragem o primeiro passo, onde valores discretos são capturados de um sinal originalmente contínuo no tempo.

A amostragem é crucial porque é nesse estágio que o sinal contínuo é transformado em uma sequência de pontos discretos que representará o sinal digitalmente. A qualidade dessa conversão afeta diretamente a precisão com que o sinal original pode ser reconstruído, evitando perdas de informação causadas por fenômenos como o \textit{aliasing}, que ocorre quando a taxa de amostragem é inadequada.

Neste trabalho, analisamos os princípios da amostragem por três métodos: ideal, natural e \textit{flat-top} (topo plano), com base no teorema da amostragem, e implementamos simulações computacionais para ilustrar os efeitos práticos da amostragem e reconstrução de sinais digitais. A simulação, baseada na modulação por amplitude de pulso (PAM) de um sinal senoidal puro, incluiu a análise do espectro de frequências, a aplicação de métodos de filtragem e a reconstrução do sinal original. Utilizamos a Transformada de Fourier (FT) para análise espectral, o teorema de Nyquist para determinar a taxa de amostragem adequada, e filtros passa-baixa para a reconstrução do sinal. O documento descreve detalhadamente cada etapa do processo, os teoremas aplicados, os cálculos realizados e as ferramentas computacionais utilizadas, além de apresentar os resultados obtidos.


\section{Fundamentação teórica}
% Aqui entra a pasta com o conteúdo de como a análise analítica foi realizada

$$
    n = \left[ \frac{\log{\frac{10^{\alpha_s/10}- 1}{10^{\alpha_p/10}-1}}}{2\log{\omega_s/\omega_p}} \right]
$$

Para um filtro num sinal com:
- Banda passante $\omega_p=2600 \, \text{Hz}$
- Banda stop $\omega_s = 3700\, \text{Hz}$
- Atenuação da banda passante $\alpha_p = 3dB$
- Atenuação da banda cortante $\alpha_s = 20dB$


\subsubsection{Executando o cálculo}
Incialmente, devemos realizar o cálculo da ordem do filtro:

\begin{align} \
    n = \left[ \frac{\log{\frac{10^{20/10}- 1}{10^{3/10}-1}}}{2\log{(2\pi \cdot3700/2\pi\cdot 2600)}} \right] \\
    n = \left[ \frac{\log{\frac{100- 1}{1.99526231497-1}}}{2\log{1.42307692308}} \right]                      \\
    n = \left[ \frac{\log{99.4712635161}}{0.30645675219} \right]                                              \\
    n = \left[ \frac{1.99769763453}{0.30645675219} \right]                                                    \\
    \\
    \therefore n = 6
\end{align}


Em seguida, obtemos o valor de $A_{max}$ para calcular a função de transferência do filtro.

$$
    H(j\omega) = \frac{1}{\sqrt{1+\epsilon^2 \left( \frac{\omega}{\omega_p} \right)^{2n}}}
$$

Como na fórmula, é preciso o valor de $\epsilon$, que é calculado da seguinte forma:

$$
    \epsilon = \sqrt{10^{A_{max}/10}-1}
$$

Considerando que o nosso $A_{max}=3\, \text{dB}$, temos que:

\begin{align} \
    \epsilon = \sqrt{10^{0.3}-1}    \\
    \epsilon = \sqrt{0.99526231497} \\
    \therefore \epsilon = 0.99762834511
\end{align}

Aplicando na fórmula da função de transferência:

\begin{align} \
    H(j\omega) = \frac{1}{\sqrt{1+0.9976^2 \left( \frac{\omega}{2\pi \cdot 2600} \right)^{2\cdot 6}}} \\
    H(j\omega) = \frac{1}{\sqrt{1+0.9953 \left( \frac{\omega}{2\pi \cdot 2600} \right)^{12}}}
\end{align} \\


\section{Resultados e discussões}
Nesta seção, apresentamos os resultados do nosso experimento de amostragem de sinais. A Figura~\ref{fig:input-signal}. mostra o sinal de entrada original, que serve como base para a aplicação das diferentes métodos de amostragem: ideal, natural e flat-top.

A análise deste sinal de entrada é crucial para entender a qualidade do sinal original e sua estrutura antes da modulação e amostragem. Observando a Figura~\ref{fig:input-signal}, podemos identificar as características principais do sinal que serão importantes para avaliar o impacto das técnicas de amostragem em etapas subsequentes.

Este sinal inicial será comparado com os sinais amostrados resultantes das técnicas aplicadas, permitindo uma análise detalhada das mudanças introduzidas durante o processo de amostragem e modulação.

\begin{figure}[H]
    \centering
    \includegraphics[width=1\linewidth]{03_results/octave_results/sinal_senoidal.png}
    \caption{Sinal senoidal de entrada}
    \label{fig:input-signal}
\end{figure}

\subsection{Ideal}

Na primeira parte da Figura~\ref{fig:ideal-pam}, observamos o sinal de trem de impulsos gerados, apresentados de forma discreto. O trem de impulsos é gerado para modular o sinal de entrada original e possibilitar a aplicação da amostragem com método ideal. Este trem é caracterizado teoricamente como conjunto infinito de impulsos unitários, de delta de dirac ($\delta$), espaçados de uma unidade, conforme a equação  \ref{tremimpulsos}. 

Após a geração do trem de impulsos, realizamos sua multiplicação junto ao sinal original, levando ao sinal como ilustrado na figura~\ref{fig:ideal-pam}, com \textit{aliasing} e na figura \ref{fig:ideal-pam-aliasing}, sem \textit{aliasing}. Esse processo resulta em um sinal modulado que reflete a amostragem do sinal original em intervalos discretos. 

Para realizar a execução das amostragens, foi necessário realizar a definição de uma frequência de amostragem, a qual é necessária para cada um dos métodos de amostragem apresentados conseguirem obter os valores corretos. Para a definição da frequência de amostragem, foi levada em consideração os valores retornados pelo código, dentre eles, o principal utilizado foi o valor da frequência máxima utilizada. Sendo $f_{max} = 910Hz; $
$F_s Nyquist = 2002Hz;$
$Aliasing F_s = 819Hz.$

\begin{figure}[H]
    \centering
    \includegraphics[width=0.8\linewidth]{03_results/octave_results/ideal_sampling.png}
    \caption{Saída sem Aliasing da amostragem Ideal}
    \label{fig:ideal-pam}
\end{figure}

\begin{figure}[H]
    \centering
    \includegraphics[width=0.8\linewidth]{03_results/octave_results/ideal_aliasing_sampling.png}
    \caption{Saída da amostragem Ideal - Frequencia de Amostragem = 819}
    \label{fig:ideal-pam-aliasing}
\end{figure}

\subsection{Natural}

Na Figura~\ref{fig:natural-pam}, observamos na primeira parte o trem de pulsos retangulares. Após gerado esse trem de pulsos, esse sinal foi multiplicado ao sinal original, gerando o sinal amostrado pelo método natural, como é possível observar na Figura~\ref{fig:natural-pam}. A amostragem onde acontece aliasing é possível observar na Figura~\ref{fig:natural-pam-aliasing}.

\begin{figure}[H]
    \centering
    \includegraphics[width=0.8\linewidth]{03_results/octave_results/natural_sampling.png}
    \caption{Saída sem Aliasing da amostragem Natural}
    \label{fig:natural-pam}
\end{figure}

\begin{figure}[H]
    \centering
    \includegraphics[width=0.8\linewidth]{03_results/octave_results/natural_aliasing_sampling.png}
    \caption{Saída da amostragem Natural - Frequencia de Amostragem = 819}
    \label{fig:natural-pam-aliasing}
\end{figure}

\subsection{Flat-Top}

Assim como na amostragem natural, é utilizado o trem de pulsos para realizar a modulação do sinal, como observado na Figura~\ref{fig:flattop-pam}. Mas, diferente da abordagem natural, no método de topo plano o trem de pulsos, quando modulado,  manterá o valor da amostragem até o próximo intervalo de amostragem $\tau$, conforme descrito no desenvolvimento das equações de \ref{eq:flattopFT} a  \ref{eq:flattop}. Os valores obtidos quando se tem \textit{aliasing} é possível observar na Figura~\ref{fig:flattop-pam-aliasing}.
\begin{figure}[H]
    \centering
    \includegraphics[width=0.8\linewidth]{03_results/octave_results/flattop_sampling.png}
    \caption{Saída sem Aliasing da amostragem Flat-Top}
    \label{fig:flattop-pam}
\end{figure}

\begin{figure}[H]
    \centering
    \includegraphics[width=0.8\linewidth]{03_results/octave_results/flattop_aliasing_sampling.png}
    \caption{Saída da amostragem Flat-Top - Frequência de Amostragem = 819}
    \label{fig:flattop-pam-aliasing}
\end{figure}

Nos três métodos de amostragem, quando a frequência de amostragem possui o valor abaixo do ideal, acontece o aliasing, resultando em um sinal reconstruído que não corresponde com o sinal original, resultando em perda de informações do sinal. Essa situação podendo ser vista na Figura~\ref{fig:ideal-pam-aliasing}, Figura~\ref{fig:natural-pam-aliasing} e Figura~\ref{fig:flattop-pam-aliasing}




\section{Conclusões}
Neste estudo, foi possível observar



\begin{thebibliography}{00}
    \bibitem{b1} SANCA, A.S. TEC501 - Eletrônica para Processamento Digital de Sinais : Notas de Aula. Universidade Estadual De Feira De Santana - Departamento De Tecnologia (DTEC). 2024.

\end{thebibliography}


\appendix
\section{Detalhes Adicionais}
\section*{Detalhes de Implementação: ADC de Dupla Inclinação}
Este apêndice fornece uma visão geral dos detalhes de implementação do nosso código de simulação no Falstad para um Conversor Analógico-Digital (ADC) de Rampa Dupla.
O trecho de código abaixo mostra como implementamos o ADC no Falstad:

\begin{lstlisting}[language=Verilog]
$ 3 9.5367431640625e-7 3.3115451958692312 81 5 50 5e-11
w 192 80 192 144 0
a 192 160 368 160 8 15 -15 1000000 -2.0386609699942313e-9 0 100000
c 192 80 368 80 0 0.000001 -0.00020386813566039314 0.001
r 128 80 192 80 0 10000
a 368 176 504 176 8 15 -15 1000000 0.00020386609699942314 0 100000
w 368 80 368 160 0
g 192 176 192 192 0 0
R 64 64 16 64 0 1 1 2.5 2.5 0 0.5
R 64 96 0 96 0 0 40 -5 0 0 0.5
150 576 160 704 160 0 2 0 5
164 856 160 976 160 0 7 0 0 0 0 0 0 0 false 0
160 128 80 64 80 0 20 10000000000
g 368 192 368 208 0 0
w 504 176 576 176 0
R 576 144 528 144 1 2 32768 2.5 2.5 0 0.1
w 96 16 96 96 0
w 952 160 952 16 0
w 1088 424 1048 424 0
166 952 424 1184 424 1024 6
w 1192 352 1168 352 0
w 1192 320 1152 320 0
w 1192 288 1136 288 0
w 1192 256 1120 256 0
w 1192 224 1104 224 0
w 1192 192 1088 192 0
w 1168 584 1168 352 0
w 1048 584 1168 584 0
w 1152 552 1152 320 0
w 1048 552 1152 552 0
w 1136 520 1136 288 0
w 1048 520 1136 520 0
w 1120 488 1120 256 0
w 1048 488 1120 488 0
w 1104 456 1104 224 0
w 1048 456 1104 456 0
w 1088 424 1088 192 0
w 1048 352 1168 352 0
w 1048 320 1152 320 0
w 1136 288 1048 288 0
w 1120 256 1048 256 0
w 1104 224 1048 224 0
w 1048 192 1088 192 0
168 952 192 1080 192 2 6 5 5 5 5 5 5
419 1192 192 1272 192 0 6
w 64 96 64 408 0
w 64 584 952 584 0
I 64 408 64 464 0 0.5 5
w 64 464 64 584 0
O 952 424 912 424 0 0
w 704 160 856 160 0
w 952 16 96 16 0
w 800 384 800 352 0
w 952 384 800 384 0
w 800 352 856 352 0
w 800 216 800 352 0
w 576 216 576 176 0
I 576 216 800 216 0 0.5 5
o 5 16 0 4099 1.25 0.00078125 0 1 Output\sIntegrator
o 7 128 0 4098 5 0.00078125 1 4 7 3 3 3 48 0
\end{lstlisting}


Como pode ser observado, o código utiliza diversos componentes do Falstad para criar o circuito do ADC.

Ao examinar esses componentes e suas interações, é possível obter uma compreensão mais profunda de como o ADC de Rampa Dupla funciona e como foi implementado no Falstad.


\end{document}
