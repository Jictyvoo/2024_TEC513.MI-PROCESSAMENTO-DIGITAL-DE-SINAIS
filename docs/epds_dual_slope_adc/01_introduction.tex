Em termos físicos e naturais, as grandezas variam de forma contínua e analógica, onde cada instante de tempo infinitesimal pode apresentar um valor que varie. Hoje, sistemas automatizados coletam dados externos analógicos, como pressão em reservatórios, temperatura em caldeiras, sinais de áudio, dentre outros.

Por outro lado, os sistemas digitais operam com sinais quantizados. Isso significa que, para processar informações analógicas, é frequentemente necessário convertê-las em um formato compatível com o processamento digital. Para esse propósito, são empregados conversores analógico-digitais (também conhecidos como A/D ou ADC), cuja compreensão das características e funcionalidades é essencial.

Portanto, entender as características dos conversores A/D e como os mesmos funcionam, se torna fundamental. Neste trabalho, iremos nos concentrar na análise da função e características dos conversores A/D de Rampa Dupla.
